\chapter{Andreev Bound States in SNS-junction}
We will in this chapter find the energies of the Andreev Bound States in an SNS-junction without any external field. blablabla
\\
\\
We consider a one-dimensional SNS-junction parallel to the $x$-axis, with $x=0$ and $x=L$ at interface between the normal metal and the leftmost (L) and rightmost (R) superconductor, respectively. We use the position-space representation $\Psi(x) = \big(\vec{u}(x) \ \ \vec{v}(x) \big)^T$ as used in equation \eqref{BdG-2}. We consider s-wave superconductors such that the gap parameter, $\Delta(x)$, is constant in each superconductor, with equal magnitude, $\Delta_0$, but allow for different phases, $\phi_L$ and $\phi_R$. Necessarily, the gap parameter is zero in the normal metal. The overall gap parameter is
\begin{equation}
    \Delta(x) = \Delta_0\left( e^{i\phi_L}\Theta(-x) + e^{i\phi_R}\Theta(x-L)\right),
\end{equation}
where $\Theta(x)$ is the Heaviside step function. 
The Hamiltonian of the system will be on the form given in equation \eqref{Ham-2-4}. We allow for different chemical potential, $\mu_S$ and $\mu_N$, and different effective mass, $m_S$ and $m_N$, in the superconductors and the normal metal, respectively. Moreover, we let $V(x)$ be a delta-potential barrier at the interfaces and allow for different strength, i.e. $V(x) = V_L\delta(x) + V_R\delta(x-L)$. The overall Hamiltonian is then
\begin{equation}
    h(x) = h_S(x)\big(\Theta(-x) + \Theta(x-L) \big) + h_N(x)\Theta(x)\Theta(L-x) + V_L\delta(x) + V_R\delta(x-L)
\end{equation}
where $h_S(x) = -\frac{\hbar^2}{2m_S}\frac{d^2}{dx^2}-\mu_S$ and $h_N(x) = -\frac{\hbar^2}{2m_N}\frac{d^2}{dx^2}-\mu_N$. The Hamiltonian, $h(x)$, gap parameter, $\Delta(x)$ and wave function, $\Psi(x)$, must satisfy the BdG-equations \eqref{BdG-2}.
\\
\\
Charge conservation yields the boundary condition
\begin{equation}
\begin{split}
    \Psi_L(0) = \Psi_N(0) \equiv \Psi(0),\\
    \Psi_R(L) = \Psi_N(L) \equiv \Psi(L).
\end{split}
\end{equation}
Since we have a delta potential at the interfaces we will not have conservation of the derivative of the wave vectors and integrate the BdG-equations \eqref{BdG-2} in order to find the boundary conditions for the derivatives:
\begin{equation*}
\begin{split}
    0 &= \lim_{\epsilon \rightarrow 0} \int_{-\epsilon}^{\epsilon} E\Psi(x) dx = \lim_{\epsilon \rightarrow 0} \int_{-\epsilon}^{\epsilon}
    \begin{pmatrix}
    \hat{H}(x) & \hat{\Delta}(x) \\
    \hat{\Delta}^{\dagger}(x) & -\hat{H}(x)
    \end{pmatrix}
    \Psi(x)dx
    \\
    &=\lim_{\epsilon \rightarrow 0} \int_{-\epsilon}^{0^-}
    \begin{pmatrix}
    h_S(x)\hat{\sigma}_0 & i\Delta(x)\hat{\sigma}_2 \\
    -i\Delta^*(x)\hat{\sigma}_2 & -h_S(x)\hat{\sigma}_0
    \end{pmatrix}
    \Psi_{L}(x)dx
    +
    \begin{pmatrix}
    V_L\hat{\sigma}_0 & 0\\
    0 & -V_L\hat{\sigma}_0
    \end{pmatrix}
    \Psi(0)
    \\
    &+\lim_{\epsilon \rightarrow 0} \int_{0^+}^{\epsilon}
    \begin{pmatrix}
    h_N(x)\hat{\sigma}_0 & 0 \\
    0 & -h_S(x)\hat{\sigma}_0
    \end{pmatrix}
    \Psi_{N}(x)dx\\
    &=
    \begin{pmatrix}
    \hat{\sigma}_0 & 0\\
    0 & -\hat{\sigma}_0
    \end{pmatrix}
    \left(
    V_L\Psi(0)
    -\frac{\hbar^2}{2}
    \lim_{\epsilon \rightarrow 0}
    \left( \frac{1}{m_S}\int_{-\epsilon}^{0^-} 
    \frac{d^2}{dx^2} \Psi_L(x) dx
    + \frac{1}{m_N}\int_{0^+}^{\epsilon} 
    \frac{d^2}{dx^2} \Psi_N(x) dx \right)
    \right)\\
    &=
    \begin{pmatrix}
    \hat{\sigma}_0 & 0\\
    0 & -\hat{\sigma}_0
    \end{pmatrix}
    \left(
    V_L\Psi(0) +
    \frac{\hbar^2}{2m_S}
    \Psi'_L(0)
    -  \frac{\hbar^2}{2m_N}
    \Psi'_N(0) 
    \right),
\end{split}
\end{equation*}
and the boundary condition at the $x=0$ interface for the derivatives is thus
\begin{equation}
    \frac{\hbar^2}{2m_N} \Psi'_N(0) - \frac{\hbar^2}{2m_S} \Psi'_L(0) = V_L\Psi(0).
\end{equation}
Similarly, we get
\begin{equation}
    \frac{\hbar^2}{2m_S} \Psi'_R(L) - \frac{\hbar^2}{2m_N} \Psi'_N(L) = V_R\Psi(L)
\end{equation}
as boundary condition at the $x=L$ interface.
\\
\\
The solution of the BdG-equations \eqref{BdG-2} that satisfy the boundary conditions will be on the form 
\begin{equation}
    \Psi_{k}(x) = 
    \begin{pmatrix}
    \vec{u}_k \\ \vec{v}_k
    \end{pmatrix}
    e^{ikx}
\end{equation}
and 
\begin{equation}
    h_{S/N}(x)\Psi_{k,S/N}(x) = \left(\frac{\hbar^2k^2}{2m_{S/N}}-\mu_{S/N}\right)\Psi_{k,S/N}(x) \equiv \varepsilon_{k,S/N} \Psi_{k,S/N}(x).
\end{equation}
We will first consider the superconducting region.We find the eigenvalues, $E_k$ of an equation of the form $A\Psi_k(x) = E_k\Psi_k(x)$, for a matrix $A$ by setting the determinant of the matrix $(A-E_kI)$ to zero. For the eigenvalue problem in the BdG-equations \eqref{BdG-2} we must for the superconducting region calculate
\begin{equation}
\begin{split}
    0 &= 
    \det\begin{pmatrix}
        \left(\varepsilon_{k,S} - E_{k,S}\right)\hat{\sigma}_0 & i\Delta\hat{\sigma}_2 \\
        -i\Delta^*\hat{\sigma}_2 & \left(-\varepsilon_{k,S} - E_{k,S}\right)\hat{\sigma}_0
    \end{pmatrix}
    \\ &= 
    \det\begin{pmatrix}(\varepsilon_{k,S} -E_{k,S})\hat{\sigma}_0\end{pmatrix}
    \det\begin{pmatrix}
    -(\varepsilon_{k,S} + E_{k,S})\hat{\sigma}_0 + i\Delta^*\hat{\sigma}_2\frac{1}{\varepsilon_{k,S}-E_{k,S}}i\Delta\hat{\sigma}_2
    \end{pmatrix}
    \\&=
    \left(\varepsilon_{k,S}-E_{k,S}\right)^2\left(\varepsilon_{k,S} + E_{k,S} + \frac{\Delta_0^2}{\varepsilon_{k,S} - E_{k,S}}\right)^2,
\end{split}
\label{det}
\end{equation}
where we have used the relation 
\begin{equation}
\det\begin{pmatrix}
A & B \\ C & D
\end{pmatrix}
= \det\big(A\big)\det\big(D-CA^{-1}B\big).
\end{equation}
As $E_{k,S} = \varepsilon_{k,S}$ would give zero in the denominator, the only solution to the above equation \eqref{det} is
\begin{equation}
E_{k,S}^2 = \varepsilon_{k,S}^2 + \Delta_0^2,
\end{equation}
whch is of the same form as equation \eqref{Ek}. And has the fourfolded degenery of relevant states:
\begin{equation}
k^{\pm} = k_S \sqrt{1 \pm \frac{\sqrt{E_{k,S}^2 - \Delta_0^2}}{\mu_S}}
\end{equation}
with $k_S = \sqrt{2m\mu_S}/\hbar$ as the Fermi wave vector in the superconductor. 
\\
\\
We find the wave functions corresponding to the eigenvalue $+E_{k,S}$, by reducing the matrix
\begin{equation}
\begin{pmatrix}
(\varepsilon_{k,S} - E_{k,S}) \hat{\sigma}_0 & i\Delta\hat{\sigma}_2 \\
-i\Delta^*\hat{\sigma}_2 & -(\varepsilon_{k,S} + E_{k,S})\hat{\sigma_0}  
\end{pmatrix}
\sim
\begin{pmatrix}
\hat{\sigma}_0 & -i\frac{u_0}{v_0}e^{i\gamma}\hat{\sigma}_2 \\
0 & 0
\end{pmatrix}
\end{equation}
which give the solutions
\begin{equation}
    \vec{u}_k = u_0e^{i\alpha}\hat{\sigma}_0
    \quad \mathrm{and} \quad
    \vec{v}_k = -iv_0e^{i\beta}\hat{\sigma}_2
\end{equation}
Similarly, the wave functions corresponding to the eigenvalue $-E_{k,S}$ give
\begin{equation}
\begin{pmatrix}
(\varepsilon_{k,S} + E_{k,S}) \hat{\sigma}_0 & i\Delta\hat{\sigma}_2 \\
-i\Delta^*\hat{\sigma}_2 & -(\varepsilon_{k,S} - E_{k,S})\hat{\sigma_0}  
\end{pmatrix}
\sim
\begin{pmatrix}
\hat{\sigma}_0 & i\frac{v_0}{u_0}e^{i\gamma}\hat{\sigma}_2 \\
0 & 0
\end{pmatrix}
\end{equation}
which give the solutions
\begin{equation}
    \vec{u}_k = -iv_0e^{-i\beta}\hat{\sigma}_2
    \quad \mathrm{and} \quad \vec{v}_k = u_0e^{-i\alpha}\hat{\sigma}_0.
\end{equation}
where we have used the relation from equation \eqref{uv-2}. The wave functions corresponding to the eigenvalues $+E_{k,S}$ and $-E_{k,S}$ are thus
\begin{equation}
\Psi_{\pm   k^{+}}(x) = 
\begin{pmatrix}
u_0e^{i\alpha}\hat{\sigma}_0 \\
-iv_0e^{i\beta}\hat{\sigma}_2
\end{pmatrix}e^{\pm ik^{+}x},
\quad \mathrm{and} \quad
\Psi_{\pm k^{-}}(x) = 
\begin{pmatrix}
-iv_0e^{-i\beta}\hat{\sigma}_2 \\
u_0e^{-i\alpha}\hat{\sigma}_0
\end{pmatrix}e^{\pm ik^{-}x},
\end{equation}
respectively.
\\
\\
In the normal region the gap parameter, $\Delta(x)$, is zero, and so the BdG-equations \eqref{BdG-2} takes the form
\begin{equation}
\begin{pmatrix}
\varepsilon_{k,N}\hat{\sigma}_0 & 0\\
0& -\varepsilon_{k,N}\hat{\sigma}_0
\end{pmatrix}\Psi_k(x)
= E_{k,N}\Psi_k(x)
\end{equation}
which give the eigenvalues $E_{k,N} = \pm \varepsilon_{k,N}$ and corresponding eigenvectors
\begin{equation}
\Psi_{\pm k^+}(x) =
\begin{pmatrix}
e^{i\alpha}\hat{\sigma}_0 \\ 0
\end{pmatrix}
\quad \mathrm{and} \quad
\Psi_{\pm k^-}(x) = 
\begin{pmatrix}
0 \\ e^{-i\alpha}\hat{\sigma}_0
\end{pmatrix}.
\end{equation}
We can right down the total wavefunction for each region; the left superconductor (L), the right superconductor (R) and the normal regian (N):
\begin{equation}
\begin{split}
\Psi_L(x) &= 
a_1
\begin{pmatrix}
u_0e^{i\alpha_L} \\ 0 \\ 0 \\ v_0e^{i\beta_L}
\end{pmatrix}e^{-ik^+x}
+
a_2
\begin{pmatrix}
0 \\ u_0e^{i\alpha_L} \\ -v_0e^{i\beta_L} \\ 0
\end{pmatrix}e^{-ik^+x}
+
a_3
\begin{pmatrix}
0 \\ v_0e^{-i\beta_L} \\ u_0e^{-i\alpha_L} \\ 0
\end{pmatrix}e^{ik^-x}
+
a_4
\begin{pmatrix}
-v_0e^{-i\beta_L}\\ 0 \\ 0 \\ u_0e^{-i\alpha_L}
\end{pmatrix}e^{ik^-x}
\\
\Psi_R(x) &= 
b_1
\begin{pmatrix}
u_0e^{i\alpha_R} \\ 0 \\ 0 \\ v_0e^{i\beta_R}
\end{pmatrix}e^{ik^+x}
+
b_2
\begin{pmatrix}
0 \\ u_0e^{i\alpha_R} \\ -v_0e^{i\beta_R} \\ 0
\end{pmatrix}e^{ik^+x}
+
b_3
\begin{pmatrix}
0 \\ v_0e^{-i\beta_R} \\ u_0e^{-i\alpha_R} \\ 0
\end{pmatrix}e^{-ik^-x}
+
b_4
\begin{pmatrix}
-v_0e^{-i\beta_R}\\ 0 \\ 0 \\ u_0e^{-i\alpha_R}
\end{pmatrix}e^{-ik^-x}
\\
\Psi_N(x) &= 
c_1
\begin{pmatrix}
e^{i\alpha_N} \\ 0 \\ 0 \\ 0
\end{pmatrix}e^{ik^+x}
+
c_2
\begin{pmatrix}
e^{i\alpha_N} \\ 0 \\ 0 \\ 0
\end{pmatrix}e^{-ik^+x}
+
c_3
\begin{pmatrix}
0 \\ e^{i\alpha_N} \\ 0 \\ 0
\end{pmatrix}e^{ik^+x}
+
c_4
\begin{pmatrix}
0 \\ e^{i\alpha_N} \\ 0 \\ 0
\end{pmatrix}e^{-ik^+x}
\\&+
c_5
\begin{pmatrix}
0 \\ 0 \\ e^{-i\alpha_N} \\ 0
\end{pmatrix}e^{ik^-x}
+
c_6
\begin{pmatrix}
0 \\ 0 \\ e^{-i\alpha_N} \\ 0
\end{pmatrix}e^{-ik^-x}
+
c_7
\begin{pmatrix}
0 \\ 0 \\ 0 \\ e^{-i\alpha_N}
\end{pmatrix}e^{ik^-x}
+
c_8
\begin{pmatrix}
0 \\ 0 \\ 0 \\ e^{-i\alpha_N}
\end{pmatrix}e^{-ik^-x}
\end{split}
\end{equation}